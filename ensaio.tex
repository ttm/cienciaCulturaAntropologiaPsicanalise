%%%%%%%%%%%%%%%%%%%%%%%%%%%%%%%%%%%%%%%%%
% Thin Sectioned Essay
% LaTeX Template
% Version 1.0 (3/8/13)
%
% This template has been downloaded from:
% http://www.LaTeXTemplates.com
%
% Original Author:
% Nicolas Diaz (nsdiaz@uc.cl) with extensive modifications by:
% Vel (vel@latextemplates.com)
%
% License:
% CC BY-NC-SA 3.0 (http://creativecommons.org/licenses/by-nc-sa/3.0/)
%
%%%%%%%%%%%%%%%%%%%%%%%%%%%%%%%%%%%%%%%%%

%----------------------------------------------------------------------------------------
%   PACKAGES AND OTHER DOCUMENT CONFIGURATIONS
%----------------------------------------------------------------------------------------

\documentclass[a4paper, 12pt]{article} % Font size (can be 10pt, 11pt or 12pt) and paper size (remove a4paper for US letter paper)
\usepackage[textwidth=12cm]{geometry}
\usepackage{hyperref}
\usepackage[english,portuguese]{babel}
\usepackage[utf8]{inputenc}
\usepackage{float}
\usepackage{marginnote}
%\usepackage{fixme}
\usepackage{fixme}
\newcommand{\ftnt}[1]{\footnote{#1 \emph{Nota d@s editoræs.}}}

\usepackage{color} % for the notes
\usepackage{xcolor}
\usepackage[protrusion=true,expansion=true]{microtype} % Better typography
\usepackage{graphicx} % Required for including pictures
\usepackage{wrapfig} % Allows in-line images
\usepackage{tocloft}
\usepackage{multirow}

\usepackage{mathpazo} % Use the Palatino font
\usepackage[T1]{fontenc} % Required for accented characters
\linespread{1.15} % Change line spacing here, Palatino benefits from a slight increase by default
\usepackage{etoolbox}
\newcommand{\firefox}{\textsc{f}irefox}
\newcommand{\floss}{\textsc{floss}}
\newcommand{\openoffice}{\textsc{o}pen\textsc{o}ffice}
\newcommand{\puredata}{\textsc{p}uredata}
\newcommand{\wiki}{\textsc{w}iki}
\newcommand{\etherpad}{\textsc{e}therpad}
\newcommand{\irc}{\textsc{irc}}
\newcommand{\ocd}{\textsc{ocd}}
\newcommand{\participa}{\textsc{p}articipa.br}
\newcommand{\httpb}{\textsc{http}}
\newcommand{\html}{\textsc{html}}
\newcommand{\nlp}{\textsc{nlp}}
\newcommand{\sectionb}{\textsc{s}ection}
\newcommand{\cn}{\textsc{cn}}
\newcommand{\aab}{\textsc{aa}}
\newcommand{\aai}{\textsc{Aa}}
\newcommand{\ontologiaa}{\textsc{o}ntologi\textsc{aa}}
\newcommand{\owl}{{\sc owl}}
\newcommand{\rdfi}{{\sc Rdf}}
\newcommand{\mongodb}{{\sc m}ongo{\sc db}}
\newcommand{\mysql}{{\sc m}y{\sc sql}}
\newcommand{\rdf}{{\sc rdf}}
%\newcommand{\paaineli}{P{\sc aa}inel}
\newcommand{\paaineli}{P{\bf \sc aa}inel}
\newcommand{\paainel}{p{\sc aa}inel}
\newcommand{\gsd}{\textsc{gsd}}
\newcommand{\ui}{\textsc{ui}}
%\newcommand{\lmb}{\url{lab\textsc{M}acambira.sf.net}}
\newcommand{\lm}{lab\textsc{M}acambira.sf.net}
%\newcommand{\lm}{\url{labMacambira.sf.net}}



\makeatletter
\renewcommand\@biblabel[1]{\textbf{#1.}} % Change the square brackets for each bibliography item from '[1]' to '1.'
\renewcommand{\@listI}{\itemsep=0pt} % Reduce the space between items in the itemize and enumerate environments and the bibliography

\hypersetup{
        colorlinks,
            linkcolor={red!50!black},
                citecolor={blue!50!black},
                    urlcolor={blue!80!black}
                }


\pretocmd{\chapter}{\addtocontents{toc}{\protect\addvspace{5\p@}}}{}{}
\pretocmd{\section}{\addtocontents{toc}{\protect\vspace{-4mm}}}{}{}
\renewcommand{\maketitle}{ % Customize the title - do not edit title and author name here, see the TITLE block below
\begin{flushright} % Right align
{\LARGE\@title} % Increase the font size of the title

\vspace{50pt} % Some vertical space between the title and author name

{\large\@author} % Author name
\\\@date % Date

\vspace{40pt} % Some vertical space between the author block and abstract
\end{flushright}
}

%----------------------------------------------------------------------------------------
%   TITLE
%----------------------------------------------------------------------------------------

\title{\textbf{Gregory Bateson}\\ % Title
%a natural collective focus\\on the collective being} % Subtitle
Antropologia,  Psicanálise, Comunicação} % Subtitle

\author{texto de \textsc{Massimo Canevacci} % Author
\\{\textit{IEA/USP, Università degli Studi di Roma La Sapienza}}} % Institution

\date{\today} % Date

%----------------------------------------------------------------------------------------

\begin{document}

\maketitle % Print the title section

%----------------------------------------------------------------------------------------
%   ABSTRACT AND KEYWORDS
%----------------------------------------------------------------------------------------

%\renewcommand{\abstractname}{Summary} % Uncomment to change the name of the abstract to something else


%\begin{abstract}
%    There are numerous pursues for a lightweight and systematic account of what is done by a group and containing individuals. The Algorithmic-Autoregulation (\aab) is a special case, in which a technical community embraced the challenge of registering their own dedication for sharing processes, self-transparency enhancements, and prove dedication. \aai\ is used since June/2011 by dozens of \floss\ and social developers, with the support of different \aab\ software gadgets and for distinct tasks. Intermittence and activity concentration of users activity follows expected natural properties. Social participation and ontological understandings of \aab\ eases comparative analysis and furthers integration.
%\end{abstract}
%\hspace*{3,6mm}\textit{Keywords:} anthropology, digital culture, , \owl, statistics, anthropological physics % Keywords

%\vspace{30pt} % Some vertical space between the abstract and first section

%----------------------------------------------------------------------------------------
%   ESSAY BODY
%----------------------------------------------------------------------------------------
\newpage
\tableofcontents


\section*{uma breve introdução}\label{sec:start}
\addcontentsline{toc}{section}{uma breve introdução}
A relação entre antropologia e psicologia (ou psicoanálise) inicia com Malinowski, fundador do método etnográfico baseado sobre a pesquisa direta no campo, observação participante\ftnt{Conceito de observação participante é ponte entre a participação social e formas de análise das participações e da própria participação do indivíduo. É um aproveitamento organico da vivêcia através da observação e participação; análise e aproveitamento.}, colher o ponto de vista nativo. Ele elabora o método do funcionalismo, através  do qual critica alguns importantes paradigmas de Marx e Freud\ftnt{Quais?}. Sobre este último, ele monstra como numa societade matrilinear, onde a transmissão geracional é  por linha materna,  a autoridade se baseia no tio materno, de consequência o complexo de Édipo no sentido universal não funciona. É  um tipo de leitura aquela freudiana relativa ao contexto europeu e a um tipo de família patrilinear e patriarcal. Nas ilhas Trobriand (onde ele fiz a celebre pesquisa) o mecanismo psicológico funciona numa maneira bem diversa. Nasce o relativismo cultural\ftnt{Conceito importante para a abertura da participação social e acolhimento de anonimato. As diferentes formas de expressar as questões e as diferentes questões envolvidas emanam tabus e atritos.}, outro conceito básico da antropologia. Cada elemento cultural precisa de ser colocado no interior specífico da sociedade e os conceitos universais, nesse sentido, são muitos problemáticos. 

Ao mesmo tempo, numa célebre fotografia, Malinowski se coloca em pé, na frente do seu “objeto” de pesquisa (um jovem trobriandês), as mãos no cinto, óculos escuro, olhar direito com que prende o dominio da situação. Com luz\ftnt{Original, ``claramente'', modificado para ``com luz'' para evitar a conotação racista da primeira expressão.}, ele nunca foi um colonialista, mas ao mesmo tempo a inteira disciplina da antropologia (e diria das ciências humanas em geral) pertenecem a um contexto histórico caracterizado pelo colonialismo. Assim, esta foto nos sugere que as dinâmicas psicológicas no interior do pesquisador são tão importantes  quanto as relações psicológicas dos sujeitos “pesquisados”. E a relação dialógica e conflitual entre os dois níveis, assim como se manifesta no fieldwork, é  parte constitutiva do processo de conhecimento. Neste campo fluido de interação híbrida se coloca a aliança entre etnografia e psicologia. A geração seguinte apresenta uma determinante mutação, assumem como figura emergente um jovem antropólogo britânico que entra em crise vertical com o funcionalismo malinowskiano hegemônico.

\section*{itinerários obliquos}\label{sec:itine}
\addcontentsline{toc}{section}{itinerários obliquos}
 O meu itinerário na antropologia cultural é oblíquo. Formei-me na Escola de Frankfurt, com dedicação particular à “dialética do Iluminismo”. Neste sentido, o conceito de cultura, no qual me iniciei, é aquele de Kultur, isto é, a cultura humanística, eurocêntrica, aquela que se inicia com a filosofia grega e chega à catástrofe dos Estados autoritários\ftnt{Vale situar melhor aqui?}. Uma cultura que tem necessidade de misturar diversas disciplinas e que apresentava então uma novidade fundamental: a reflexão filosófica aplicada na pesquisa empírica\ftnt{Antecedente academico para o Auto-Aproveitamento.}. Uma filosofia social cujo telos – o escopo final – consistia em transformar o mundo segundo a célebre XI tese sobre Feuerbach\ftnt{``Os filósofos têm apenas interpretado o mundo de maneiras diferentes; a questão, porém, é transformá-lo.''}, de Marx. Depois, por um acaso, logo que me formei, o professor de Antropologia Cultural me chamou para colaborar na Faculdade de Sociologia, pois queria conhecer a nossa cultura antes de estudar a cultura dos “outros”. Nesse começo, e devido a um novo acaso, fui convidado a ensinar no Brasil, em 1984, e o meu ponto de vista começou a mudar profundamente. Eu descentralizei a grande cultura ocidental como uma das culturas e filosofias possíveis. Deixei dolorosamente, diria traumaticamente, a minha formação clássica: foi um presente precioso que o Brasil me deu. Assim dei início a uma pesquisa espontânea, e depois mais metodológica, sobre São Paulo. 

Sempre tive uma paixão irrefreável pelo cinema em particular e pela comunicação e as artes visuais em geral. Por isso, decidi realizar, fazendo uso de diversos métodos, uma pesquisa empírica sobre a comunicação visual acerca da metrópole de São Paulo. Utilizei para isso o conceito de polifonia, que integrei ao título final de minha pesquisa: A cidade polifônica~\cite{b1} – uma miscelânea de escrita ensaísta, narrativa, etnopoética e imagens. Comecei essa pesquisa fotografando alguns lugares de São Paulo, seguindo a hipótese de quatro centros: a Faria Lima chegando à avenida Berrini era uma possibilidade distante e de interconexão necessária e, para mim, ali emergia outro centro de estilo pós-industrial. Depois de fotografar alegorias, estátuas de pedra, seringueiras, trabalhadores da construção suspensos em andaimes, evangélicos pregando na rua, elegi os grandes edifícios modernistas, aqueles de Lina Bo Bardi, que amo, desmistificando a pirâmide da Fiesp na Paulista. Em suma, os trabalhos sobre e com as imagens eram dialógicos com a escrita. 

Posteriormente, e de novo por acaso, encontrei um cacique xavante – Domingos Mahoro’e’o –, que me convidou para visitar sua aldeia. Então, finalmente comecei a fazer pesquisas indígenas no Mato Grosso, entre os Xavantes e depois entre os Bororos. A participação nos rituais xavantes de furação das orelhas e no funeral bororo foram as experiências da minha vida. As imagens foram sempre decisivas, mas, para minha grande surpresa, no começo eram como um desafio e se transformaram em um prazer. Entre essas duas culturas, havia pessoas como Divino (xavante) e Paulinho (bororo) que usaram o vídeo. Daí a minha posição atual com base na autorrepresentação~\cite{b2}, ou melhor, uma tensão dialógica e até conflitos entre auto e heterorrepresentação\ftnt{Representacao de si e do que é diferente de si?}. No fim de meu atual projeto, o pressuposto que considero fundamental para muitos pontos de vista é a relação aldeia-metrópole. Ou seja, \emph{uma etnografia que transita entre as culturas indígenas e urbanas para encontrar pontos de contato ou de diferença, de conflito, de sincretismo cultural}\ftnt{}. Assim, comunicação-cultura-consumo\ftnt{Conceituação-chave para transições atuais.} desempenham um papel sempre mais importante na metrópole contemporânea e, simetricamente, o conceito de moderno está em evidente declínio.
\nocite{b3}


\section*{Gregory Bateson: etnógrafo da complexidade}\label{sec:bat}
\addcontentsline{toc}{section}{Gregory Bateson: etnógrafo da complexidade}

Por, meu trabalho é diretamente influenciado por Gregory Bateson. Admiro-o e, ao mesmo tempo, tento identificar algumas limitações em seu contexto histórico e cultural. O livro de autoria de Bateson que mais me impressionou foi Balinese Character~\cite{b4} – na minha opinião, a melhor pesquisa etnográfica já realizada com uma câmara de filmar e fotografar. Insuperável. O conceito de uma sequência que define um traço cultural (por exemplo, o aleitamento ou o transe) constitui a base para minha pesquisa e meu ensino. Sempre que o mostro a sala de aula forma-se um silêncio atento para o processo de investigação, ponto de partida para o desenvolvimento posterior de conceitos fundamentais, como o duplo vínculo (double bind) e a ecologia da mente\ftnt{Procurarei melhor sobre estes dois conceitos.}. O primeiro conceito – o duplo vínculo – foi especialmente aplicado à comunicação visual, por meio da publicidade, da internet, do cinema e da política. Trata-se de um conceito que perpassa a psicologia, a etnografia, a comunicação, com um projeto de libertação. Todos os alunos (inclusive eu) estão cheios de duplos vínculos. Fixá-los e tentar dissolvê-los criativamente é a grande lição de Gregory Bateson que tento aplicar nos fetichismos visuais atuais.

%\marginpar{Owww yeah} 
Já Ecologia da mente~\cite{b5} é um conceito mais articulado: há muitas limitações genéricas que se tornam estilos comuns, como o filme Avatar (2009), de James Cameron, no qual alguns críticos (e não só) conseguiram enxergar algo de Bateson. E, talvez, estejam certos, o que se deve também a ele. A trama que liga (patterns which connects) é sem dúvida importante, embora descambe facilmente para um hippie místico zen, trip-ayuasca, uga-uga e coisas do tipo. Isso me deixa desconfiado em relação ao seu conceito de holístico, que considero perigosíssimo: a totalidade inclui e explica uma parte ou os diversos elementos empíricos. Em todo caso, reivindico a subjetividade - de um novo tipo a que chamo de multivíduo -  como não unificável em uma totalidade ecológica. Este é um erro de Bateson: a ansiedade de perder a si mesmo ou unificá-lo holisticamente com o todo\ftnt{Importante em relação a critica feita no ultimo encontro. Bom esse ponto, pois ja antecipa a divergencia em relação ao Bateson.Tendo a concordar com Bateson, mas tenho que ler melhor o autor. Achei o contraponto da visão valiosíssimo para estabelecer/contextualizar o vetor/dipolo holístico $\rightarrow$ multivíduo.}.
%\marginnote{typeset text here...}[2cm] 

Bateson me influenciou na percepção da ligação entre etnografia e cultura digital\ftnt{Encontrei este livro que parece elucidador, mas apenas comecei a ler. Guardo aqui como material de pesquisa: 
\url{http://www.livroslabcom.ubi.pt/pdfs/20110819-centeno_maria_joao_conceito_de_comunicacao.pdf}. Boa próxima leitura.}: a sua participação no nascimento da cibernética com Wiener foi muito importante. Daí a minha pesquisa sobre a internet e o sincretismo digital. Queria sublinhar que a cultura digital tem uma história que sempre esteve interligada à antropologia. E o autor de referência nessa conexão é Bateson.\ftnt{Informação básica, para manter em mente.}.

Na entrevista realizada por Steward Brand, publicada em Per l’amor di Dio, Margaret!~\cite{b6}, Bateson revela a escolha de colaborar com o fundador da cibernética, Nobert Wiener, no ano de 1946, quando então abandonou a “disciplina” por incluir a cibernética na área da antropologia. 

Bateson já tinha elaborado nos anos  ’30 o conceito de schismogenesis (cismogenesis do grego: schisma = divisão +  genesis = nascimento) durante suas primeiras pesquisas etnográficas em Nova Guiné; para ele,  os processos comportamentais e interativos no interior de um grupo em relaçao ao ethos (como uma cultura enfrenta e resolve as emoções) podem favorecer seja competição ou rivalidade e seja inibição ou submição. Ambas  podem ser autodestrutivas por duas facções internas ao grupo ou resolver-se numa divisão mais ou menos dramática: por isso, se criam mecanismos de autocorreção que freiam as relações de tipo conflitual. Quero sublinhar a importância não só conceitual mas tambén pragmática da relação entre schismogenesis e autocorreção por este motivo: dez anos depois e por outros itinerários epistemológicos, Norbert Wiener elabora o modelo de retroação – o feedback – como afim ao modelo de autocorreção cismogenética. Tudo isso significa uma aliança profunda (ou conexões psicoculturais) entre feedback e schismogenesis em direção de verificar como as tecnologias podem ser aplicadas na criação projetual da primeira inteligência artificial. E justamente a cibernética nasce no encontro entre um pesquisador etnográfico isolado (Bateson) e uma equipe de pesquisadores informáticos (Wiener). Tal aliança entre as chamadas duas culturas (científica e humanística segundo Egdar Snow) torna-se ainda mais significativa, enquanto Bateson e Wiener criticam os cientistas que isolam o input-output sem reatroação, analizando o “objeto” enquanto ficam fora dele. Wiener e Bateson utilizam a metafora da caixa (box): o cientísta precisa ficar dentro da caixa, isto é, fora da metáfora, no interior do fieldwork etnográfico. E este fieldwork presenta afinidades (não identitade!), entre o ethos do Iatmul na Nova Guiné e  a inteligência artificial na cibernética, baseadas sobre autocorreção. Quero sublinhar de novo como estas metodologias etnográficas  são muito similares àquela da psicanálise. O/a psicanalista precisa ficar dentro da relação com o paciente, não pode ficar observando-o ou escutando-o de fora. A caixa è também o set psicanalítico onde se cria uma contínua retroação autocorretiva entre os dois sujeitos involvidos. O feedback involve o psicanalista o paciente e as metodologias psicanalíticas traduzem este feedback nos seus próprios conceitos (i.e. transfer e contra-transfer). Recíprocas autocorreções criam um complexo vínculo entre os dois durante a “interminável” terapia e assim se apresenta o novo conceito elaborado mais recentemente onde o pensamento de Bateson virou fundamental: aquele de complexidade.

Von Foerster explica: “O que se precisa agora é uma descrição do descritor; ou, em outras palavras, precisamos de uma teoria do observador” (em Brand, 2004:152); isso significa uma aliança necessária ainda mais profunda e complexa entre antropologia e psicanálise, no contexto do desafio  transdisciplinar da complexidade. Descrever, interpretar e transformar o descritor. E se um cientista “exato” fala assim, me parece que esta aliança está já profunda e praticada. O que ainda hoje precisa de ser colocado na caixa é a descrição do observador, tanto etnógrafo como psicanalista ou epistemólogo: isto pra mim significa aprender a fazer pesquisa com (e não sobre) os sujeitos envolvidos no processo empírico, seja nativo, paciente ou um sincrotone\ftnt{O que é?}. A “descrição do descritor” é uma mudança de cultura epistemológica, comunicacional e política que envolve no processo compositivo ou terapêutico cada sujeito da pesquisa. 

Tudo isso precisa de ser aplicado também na cultura digital.  Nas conexões  e infunções\ftnt{Infunções são desfunções?} entre etnografia, psicanálise e cibernética – uma psico-etnografia da web – nascem as possibilidades de mudar a internet e uma “coisa” ainda mais ampla: a composição do sujeito contemporâneo. Assim, agora se apresenta o problema não resolvido – político e epistêmico – dos softwares produzidos como resultado de elaborações informáticas. Uma nova elaboração de software não baseada sobre a lógica binária poderia ser produzida a partir desta aliança da complexidade trandisciplinar. E o digital cruza etnografia, psicanálise e comunicação.\ftnt{O parágrafo do texto, IMHO.}

Por isso, precisamos que as disciplinas se conectam, por meio dos fluxos da comunicação digital, cuja aliança com os profissionais da informática – frequentemente fechados como muitos cientistas sociais em mundos encastelados – poderia favorecer soluções progressivas além da web 2.0 (importante pelo social network mas ainda centralista) em direção à web 3.0, um software  mais descentralizado e pluralista. É necessário, portanto, dissolver os poderes econômicos da web 2.0; envolver cada cibernauta nos procesos elaborativos multilógicos e multissensoriais; favorecer um proceso de autopoies por cada sujeito multividual. A aliança entre etnografia e psicanálise atual precisa de enfrentar também – as vezes principalmente  - esta nova composição do multivíduo digital, entre novas patologias e inovações comunicacionais libertadoras.

Enfim, eu li Naven~\cite{b7}, outro livro de Bateson, em 1988 e desde o início essa obra influenciou minha cidade polifônica: sua escolha metodológica foi voltar ao mesmo ritual com pontos de vista disciplinar e óticas diferenciadas, numa diferenciação epistêmica sem-fim, enquanto um fato empírico como um ritual nunca poderia ser compreendido em sua totalidade através de um método ou uma monoescritura. Em suma, a multiplicação de pontos de vista dos pesquisadores sobre o próprio objeto de pesquisa tem sido decisiva. 

Devo dizer que, devido a isso, se desenvolveu em mim a necessidade de ver a dimensão subjetiva do objeto, para dar voz à individualidade que a antropologia cultural, mesmo que batesoniana, silencia, ignora ou até mesmo remove. Minha intenção é encontrar a individualidade no trabalho de campo, ainda que sem nome e voz. Neste sentido, o excesso de “objetivismo” ligado ao excesso de um naturalismo transcendente é o seu limite. Como já introduzi, o conceito de ethos – de que forma as emoções são produzidas, fixadas e modificadas culturalmente – é outra importante categoria aplicada à pesquisa e ao pesquisador. O estudo etnográfico das emoções e dos desvios patológicos comunicacionais é um dos grandes méritos de Bateson, fato que provocou a dura crítica de um Malinowski bloqueado no funcionalismo e que favoreceu a sucessiva pesquisa sobre o duplo vínculo e a esquizofrenia.

Naven contribuiu para a crise da objetividade na pesquisa, para a aproximação constante a um núcleo de verdade etnográfica que sempre foge, que irá se mover cada vez para mais longe, enquanto o mesmo ritual e as pessoas que o praticam mudam, assim como o sujeito que faz a pesquisa é sempre diferente.

Enfim, a etnografia é acabável e inacabável, como diria Freud para a análise\ftnt{Gostaria de mais informação sobre isso.}. Talvez se possa afirmar que Bateson se sentisse limitado pela disciplina, motivo pelo qual teria influenciado muitas pessoas que não se tornaram antropólogas no sentido restrito. Na minha experiência, posso dizer que ele me influenciou profundamente, que a leitura e a visão atenta de Naven, de Balinese character e de Ecologia da mente me formaram.

Devo mencionar também um autor contemporâneo de Bateson, totalmente diferente dele, formando um par que nunca se encontrou física ou cientificamente; refiro-me a Walter Benjamin. Aqui sinalizo outro aspecto metodológico inerente a ambos além de um certo “misticismo imanente”. Bateson diz no célebre posfácio de Naven que o método está em colocar junto os dados – o que é fundamental em toda pesquisa contemporânea, eu acho. Benjamin, mais sensível ao cinema e à tecnologia reproduzível, afirmava que o método está na montagem. Portanto, a composição é para mim o conceito mais adequado ao lugar da escrita, a fim de dar sentido à pesquisa de campo: uma montagem de fragmentos escritos, ensaísticos, literários, poéticos, icônicos, sônicos para a qual uma composiçao fluida consegue dar um sentido parcial e temporâneo\ftnt{Temporário/efêmero ou contemporâneo?}, oblíquo e profundo. 

\section*{palavras adicionais sobre o texto, sua edição e anotações}\label{sec:pal}
\addcontentsline{toc}{section}{palavras adicionais sobre o texto, sua edição e anotações}
Este texto parece ter sido escrito em 2009 ou 2010. Massimo repassou para Renato Fabbri e Marília Pisani,
 que comentaram o texto para desenvolvimentos em andamento de física antropológica, teoria crítica e cultura digital.
A edi(tora)ção final foi feita para facilitar a referência nas discussões e pesquisas.

%----------------------------------------------------------------------------------------
%   BIBLIOGRAPHY
%----------------------------------------------------------------------------------------

\bibliographystyle{unsrt}
%\bibliographystyle{plain}
%\bibliographystyle{ieeetr}
\bibliography{ensaio}

%----------------------------------------------------------------------------------------

\end{document}
